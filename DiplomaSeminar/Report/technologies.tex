\section{Wykorzystane technologie}
\subsection{Git}
\textbf{Git} to rozproszony system kontroli wersji. Zyskujemy dzięki niemu
\begin{itemize}
 \item Pełną kopię pracy na dwóch serwerach
 \item Udogodnienia dotyczące współpracy programistów \begin{itemize}
  \item Wersjonowanie kodu
  \item Łatwą wymianę kodu i pracy
  \item Możliwość rozwiązywania konfliktów
  \item Możliwość tworzenia lokalnych kopii pracy bez dostępu do internetu
  \end{itemize}
 \item Kontrolę jakości kodu znajdującego się w repozytorium
\end{itemize}

\subsection{Hudson}
Hudson to narzędzie wspomagające ciągłą integrację. W naszym przypadku, ciągła integracja oznacza przede wszystkim
\begin{itemize}
 \item Kontrola jakości kodu w skład której wchodzi: \begin{itemize}
  \item Weryfikacja poprawności składniowej kodu umieszczonego w repozytorium (automatyczne uruchomienie kompilacji)
  \item Weryfikację jakości kodu umieszczonego w repozytorium (kompletność dokumentacji, zgodność z kanonem kodowania)
  \item Uruchomienie testów \begin{itemize}
   \item Jednostkowych
   \item Funkcjonalnych
   \item Integracyjnych
   \end{itemize}
  \end{itemize}
 \item Wygenerowanie dokumentacji na podstawie adnotacji w kodzie
 \item Codzienne przygotowanie pakietu dla programistów zawierających wszystkie pliki niezbędne do rozpoczęcia współpracy z frameworkiem
\end{itemize}

\subsection{WebServices}
Mechanizm aktualizacji modułów pozwala na pobieranie ich aktualizacji z tzw. repozytorium modułów. Żeby uprościć schemat komunikacji zdecydowaliśmy, iż repozytorium będzie obsługiwało metody webservice'owe, takie jak wskazanie listy znanych modułów (i ich wersji), wystawienie archiwum z modułem, etc. Dzięki temu unikamy parsowania plików htmlowych czy xmlowych - całość realizowana jest w ramach protokołu SOAP.

\subsection{Platforma .NET}
Całość naszej pracy wykorzystuje technologię Microsoft .NET. Oznacza to w praktyce, że żeby napisać moduły do aplikacji, można posłużyć się jednym z wielu języków wspieranych przez tę platformę, m.in. C\#, C++, F\#, J\# i wiele, wiele innych. W szczególności wykorzystywane jest technologia \textbf{WPF} -- Windows Presentation Foundation, która odpowiada za zarządzanie oknami aplikacji.

Dzięki technologii WPF możemy pozwolić programiście aplikacji udostępnić fragmenty widoków bezpośrednio dla modułów. Ma na to wpływ jedna z własności tej technologii - bindingi, które umożliwiają wiązanie modelu wprost z widokiem. Model nie wie nic o swoim widoku, widok zaś posiada wiedzę na temat elementów modelu. Taki model nazywany jest w literaturze \textit{ViewModelem}.

