\section{EventAggregator}
System komunikacji międzymodułowej musi spełniać następujące założenia:
\begin{itemize}
 \item Moduł obsługuje zdarzenia zachodzące w systemie \begin{itemize}
  \item Nieznane miejsca zajścia zdarzenia -- moduł, który oczekuje na dane zdarzenie nie posiada wiedzy na temat tego, kto dostarcza wskazanego typu zdarzenia, nie ma pojęcia kiedy i gdzie ono zachodzi
  \item Moduły pojawiają się w różnym czasie -- moduł, który oczekuje na zdarzenie może zostać załadowany przed modułem, który takie zdarzenia publikuje
 \end{itemize}
 \item Moduł żąda wykonania zadania \begin{itemize}
  \item Nieznany wykonawca -- moduł oczekujący na zdarzenie nie wie, jaki moduł dostarcza zdarzenia
  \item Wielu wykonawców -- wiele modułów może oczekiwać wskazanego zdarzenia, wszyscy muszą dostać opublikowane zdarzenie w skończonym czasie
  \item Brak wykonawcy -- może nie istnieć moduł, który oczekuje zdarzenia publikowanego przez inny moduł
  \item Asynchroniczność -- publikowanie zdarzenia może odbywać się synchronicznie, bądź asynchronicznie (by nie blokować wątku publikującego)
 \end{itemize}
\end{itemize}

Powyższe wymagania wypełniać ma EventAggregator, którego zadania można określić w ten sposób:
\begin{itemize}
 \item Zapewnia komunikację (międzymodułową)
 \item Wyróżniamy dwie strony komunikacji \begin{itemize}
  \item Publikującą zdarzenie
  \item Oczekującą na zdarzenie
  \end{itemize}
 \item Różne metody komunikacji \begin{itemize}
  \item Synchroniczna -- wątek publikujący będzie czekał na wykonanie wszystkich oczekujących zdarzeń (a ich wykonanie nie będzie zrównoleglone)
  \item Asynchroniczna -- wątek publikujący po wywołaniu zdarzenia będzie kontynuowany, a obsługa zdarzeń oczekujących odbędzie się w osobnym wątku
  \item W dedykowanym wątku -- wątek publikujący będzie mógł wskazać wątek, w którym ma nastąpić obsługa zdarzeń (np. może to być wątek GUI)
 \end{itemize}
\end{itemize}

W ogólności, proces publikowania / subskrybowania odbywa się w następujących krokach:
\begin{enumerate}
 \item Dowolne moduły subskrybują się na określony \textit{typ} zdarzenia
 \item Dowolne moduły publikują określony typ zdarzenia, który jest dostarczany do wcześniej zasubskrybowanych modułów (w szczególności do żadnego)
\end{enumerate}

Przykład wykorzystania EventAggregatora z kodu modułu obrazuje kod \ref{eventaggregator}
\lstset{caption={Wykorzystanie Event Aggregatora przez moduły}}
\label{eventaggregator}
\begin{lstlisting}[frame=tb]
// subskrypcja
_eventAggregator.Subscribe<MessageType>(payload => HandlePayload);

// publikacja
_eventAggregator.Publish(sentPayload);
\end{lstlisting}

